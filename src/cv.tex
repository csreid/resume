\documentclass[11pt,letterpaper]{article}
\usepackage[margin=0.75in]{geometry}
\usepackage{enumitem}
\usepackage{titlesec}
\usepackage{xcolor}
\usepackage{etoolbox}

% Remove page numbers
\pagestyle{empty}

\setlength{\parindent}{0pt}

% Define section formatting
\titleformat{\section}{\large\bfseries\uppercase}{}{0em}{}[\titlerule]
\titlespacing{\section}{0pt}{12pt}{6pt}

% Define subsection formatting
\titleformat{\subsection}{\normalsize\bfseries}{}{0em}{}
\titlespacing{\subsection}{0pt}{8pt}{2pt}

% Remove default list formatting
\setlist[itemize]{leftmargin=0pt, itemsep=2pt, parsep=0pt, topsep=4pt}
\setlist[enumerate]{leftmargin=1.5em, itemsep=4pt, parsep=0pt, topsep=4pt}

% Contact information commands
\newcommand{\street}{5710 N Winthrop Ave}
\newcommand{\city}{Chicago}
\newcommand{\state}{IL}
\newcommand{\zip}{60660}
\newcommand{\phone}{(219) 208-0068}
\newcommand{\email}{csreid.cr@gmail.com}

\newcommand{\makeaddress}{%
	\street, \city, \state \zip \\
	\phone\ $\cdot$ \email
}

% Publication list infrastructure
\newcounter{pubcount}
\setcounter{pubcount}{0}
\newcommand{\publist}{}

% \publication command
% Usage: \publication{Authors}{Title}{Venue}{Year}{Citations}
\newcommand{\publication}[5]{%
	\stepcounter{pubcount}%
	\expandafter\gappto\expandafter\publist\expandafter{%
		\noexpand\item \textbf{#2} \\
		#1 \\
		\textit{#3}, #4 \ifx&#5&\else\ \fi
	}%
}

% \makepublications command
\newcommand{\makepublications}{%
	\begin{enumerate}[label={[\arabic*]}, leftmargin=2em]
		\publist
	\end{enumerate}
}

% Job entry command
\newcommand{\job}[4]{%
	\subsection{#1 --- #2}
	\textit{#3}

	#4

	\vspace{0.1in}
}

% Education entry command
\newcommand{\education}[4]{%
	\textbf{#1} \\
	#2 \hfill #3 \\
	#4
	\vspace{0.1in}
}

% Add publications here
\publication{C Reid, W Hafez, A Nazeri}{Mutual Information Tracks Policy Coherence in Reinforcement Learning}{arXiv preprint arXiv:2509.10423}{2025}

\publication{W Hafez, A Nazeri, C Reid, S Eshami}{Entanglement Learning: an Information-Theoretic Framework for Adaptive Convolutional Neural Networks}{2025 IEEE Conference on Artificial Intelligence (CAI)}{2025}

\publication{C Reid, S Mukhopadhyay}{Mutual reinforcement learning with heterogeneous agents}{2021 IEEE International Conference on Smart Computing (SMARTCOMP)}{2021}

\publication{C Reid, S Mukhopadhyay}{Mutual Q-learning}{2020 3rd International Conference on Control and Robots (ICCR)}{2020}

\publication{C Reid}{Mutual Reinforcement Learning}{Purdue University}{2021}

\publication{C Reid}{Student/Teacher Advising through Reward Augmentation}{arXiv preprint arXiv:2002.02938}{2020}

\begin{document}

% Name header
\begin{center}
{\huge\bfseries{Cameron Reid}}\\[0.1in]
\makeaddress
\end{center}

\section{Education}

\education{University of Illinois Chicago}{Ph.D. in Computer Science (in progress)}{2024 -- Present}{Advisor: Professor Wenhao Luo}

\education{Purdue University, Indianapolis}{M.Sc. in Computer Science}{May 2021}{Thesis: Mutual Reinforcement Learning}

\education{Purdue University, West Lafayette}{B.S. in Computer Science}{May 2014}{}

\section{Research Interests}

Mobile robotics, end-to-end deep learning control systems, multi-agent reinforcement learning, intelligent systems, machine learning for robotics

\section{Publications}

\makepublications

\section{Professional Experience}

\job{Jacobian Labs}{Founder/Principal Engineer}{June 2023 -- Present}{Founded consulting practice focused on building reliable, robust software systems for clients across diverse industries, specializing in machine learning and scalable architecture.}

\job{Klaviyo}{Senior Software Engineer}{June 2022 -- June 2023}{Led development of high-throughput data ingestion systems managing large-scale streams from e-commerce integrations, ensuring reliability and efficiency in processing millions of events.}

\job{Sense}{Senior Software Engineer}{February 2021 -- June 2022}{Guided development of conversational AI assistant using deep learning NLU techniques as part of the chatbot team, delivering Sense's primary AI product to market.}

\job{Kerauno}{Principal Software Engineer}{September 2018 -- November 2019}{Led engineering team in building next-generation communications workflow platform using modern best practices, including PostgreSQL relational data modeling and Golang microservices orchestrated with Kubernetes.}

\job{Torchlite}{Senior Platform Engineer}{December 2015 -- September 2018}{Served as founding member of engineering team, architecting and implementing all back-end infrastructure for early-stage startup.}

\job{Emerging Threats / Proofpoint}{Software Engineer}{July 2014 -- December 2015}{Developed robust big data management solutions for cybersecurity threat detection systems.}

\section{Technical Skills}

\textbf{Machine Learning:} PyTorch, TensorFlow, Keras, reinforcement learning frameworks \\[0.05in]
\textbf{Cloud \& Infrastructure:} AWS, Azure, GCP, Docker, Kubernetes \\[0.05in]
\textbf{Databases:} PostgreSQL (expert level), distributed database systems \\[0.05in]
\textbf{Programming Languages:} Python, JavaScript/TypeScript, Scala, Rust, Ruby, Go, Java

\section{Selected Projects}

\textbf{Chatbot V2:} Designed and implemented successor to Sense chatbot with improved scalability, maintainability, and stability, enabling confident deployment in demanding enterprise markets.

\vspace{0.1in}

\textbf{Mutual Reinforcement Learning:} Conducted research into novel multi-agent machine learning algorithms, resulting in multiple peer-reviewed publications.

\vspace{0.1in}

\textbf{Kerauno Event Engine:} Architected high-throughput, fault-tolerant, distributed event reactor serving as the core of Kerauno's communication orchestration platform.

\end{document}
